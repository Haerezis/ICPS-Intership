%%%%%%%%%%%%%%%%%%%%%%%%%%%%%%%%%%%%%%%%%%%%%%%%%%%%%%%%%%%%%%%%%%%%%%
% LaTeX Example: Project Report
%
% Source: http://www.howtotex.com
%
% Feel free to distribute this example, but please keep the referral
% to howtotex.com
% Date: March 2011 
% 
%%%%%%%%%%%%%%%%%%%%%%%%%%%%%%%%%%%%%%%%%%%%%%%%%%%%%%%%%%%%%%%%%%%%%%
% How to use writeLaTeX: 
%
% You edit the source code here on the left, and the preview on the
% right shows you the result within a few seconds.
%
% Bookmark this page and share the URL with your co-authors. They can
% edit at the same time!
%
% You can upload figures, bibliographies, custom classes and
% styles using the files menu.
%
% If you're new to LaTeX, the wikibook is a great place to start:
% http://en.wikibooks.org/wiki/LaTeX
%
%%%%%%%%%%%%%%%%%%%%%%%%%%%%%%%%%%%%%%%%%%%%%%%%%%%%%%%%%%%%%%%%%%%%%%
% Edit the title below to update the display in My Documents
%\title{Project Report}
%
%%% Preamble
\documentclass[paper=a4, fontsize=11pt]{scrartcl}
\usepackage[utf8]{inputenc}
\usepackage[T1]{fontenc}
\usepackage{fourier}

\usepackage[english]{babel}                                                         % English language/hyphenation
\usepackage[protrusion=true,expansion=true]{microtype}  
\usepackage{amsmath,amsfonts,amsthm} % Math packages
\usepackage[pdftex]{graphicx}   
\usepackage{url}
\usepackage{listings}
\usepackage{arydshln}

%%% Custom sectioning
\usepackage{sectsty}
\allsectionsfont{\normalfont\bfseries}
\usepackage[nottoc,notlot,notlof]{tocbibind}


%%% Custom headers/footers (fancyhdr package)
\usepackage{fancyhdr}
\pagestyle{fancyplain}
\fancyhead{}                                            % No page header
\fancyfoot[L]{}                                         % Empty 
\fancyfoot[C]{}                                         % Empty
\fancyfoot[R]{\thepage}                                 % Pagenumbering
\renewcommand{\headrulewidth}{0pt}          % Remove header underlines
\renewcommand{\footrulewidth}{0pt}              % Remove footer underlines
\setlength{\headheight}{13.6pt}

\newtheorem{defn}{Definition}[section]

%%% Equation and float numbering
\numberwithin{equation}{section}        % Equationnumbering: section.eq#
\numberwithin{figure}{section}          % Figurenumbering: section.fig#
\numberwithin{table}{section}               % Tablenumbering: section.tab#

\setcounter{secnumdepth}{4} % default value for 'report' class is "2"


%%% Maketitle metadata
\newcommand{\horrule}[1]{\rule{\linewidth}{#1}}     % Horizontal rule
\usepackage[toc]{glossaries}
%\newglossaryentry{<label>}
%{
  %name=<name>,
  %description={<description>}
  %plural={<termes>}
%}

%\newacronym{<label>}{<abbrv>}{<full>}

%\gls{<label>} %Use the entry
%\gls: singlulier sans majuscule;
%\Gls: singlulier avec majuscule;
%\glspl: pluriel sans majuscule;
%\Glspl: pluriel avec majuscule.


\newglossaryentry{self-dependency}
{
  name=self-dependency,
  description={A dependence from a statement to itself},
  plural=self-dependencies
}


\title{
        %\vspace{-1in}  
        \usefont{OT1}{bch}{b}{n}
        \normalfont \normalsize \textsc{University of Strasbourg} \\ [25pt]
        \horrule{0.5pt} \\[0.4cm]
        \huge High-Level Optimization Driven by Statement Profiling \\
        \horrule{2pt} \\[0.5cm]
}
\author{
        \normalfont                                 \Large
        Thomas Kuntz \\                                \normalsize
        Master RISE \\                                \normalsize
        UFR d'Informatique \\                                \normalsize
        Supervisor : Cédric Bastoul\\
}
\date{}
\makeglossaries

%%% Begin document
\begin{document}
%TODO Fill glossary and add link to glossary for used word

\maketitle
\thispagestyle{empty}

\clearpage

\tableofcontents
\clearpage

\section{Presentation of the laboratory and research team}
This intership falls within the research conducted in the ICPS team (Parallel and Scientific
Computing) of the ICube laboratory at the university of Strasbourg, and the CAMUS group
(Compilation for MultiCore Architectures) at Inria, the French Institute for Research
in Computer Science.

ICPS/CAMUS focuses at providing developers with theoretical and software
tools to develop efficient applications for parallel architectures without sacrificing
productivity.

Parallel systems are now omnipresent, from supercomputers to mobile devices,
and their effective use requires developers to design parallel programs or to rewrite
legacy sequential applications. However, parallel programming is still complex and out
of reach for non experts, at either designing, writing or debugging level.

To address this issue, ICPS/CAMUS is developing source-to-source compiler techniques
for automatic program optimization and parallelization. Using these technologies,
developers can continue to write sequential programs while the mapping to parallel
architecture is computed automatically.

The team has 7 faculty members and a dozen of PhD students,
postdocs or engineers. Its members are deeply involved in European and international
collaborations on scientific projects both with academia and industry.
Its most renowned technical contributions in this field include
the PolyLib library, the code generator CLooG, the dynamic optimizer VMAD or the parallelizer
of binary programs BinPar.


\section{Problematic : High-Level Optimization Driven by Statement Profiling}
Many computation-intensive programs spend a very large amount of their execution
time inside nested loops. Thus, loop nest optimization is one of the major approach used
for program optimisation.

To represent the loop nest, we can use the Polyhedral Model, a powerful mathematical model
that allows to express loop nest as union of polytopes and loop optimizations as
transformation of these polytopes.

The game is to find the "good" polytope transformations to end up with
transformed polytopes representing loop nest (which doesn't violate dependences) 
that can be parallelized, vectorized, tiled, etc. with better properties than the original
polytope.

The Polyhedral Model is really good with short and simple codes, but struggle with
large and complex codes.The reason is twofold. First, polyhedral techniques exhibit
a very high worst-case complexity which hampers its applicability to large codes.
Second, contradicting optimization objectives may result in suboptimal global
decisions


We would like to address these issues by proposing a new technique : before any
high-level optimizations are used on the source code, a first pass is applied (our pass),
aggregating "near behaving" statements together using their profile.
By doing that, we hope to ease the work of the high-level optimizer by "reducing" the size
of large and complex code, and prevent it from separating statements that should
not be separated (for example if they have a lot of data reuse in common).

"Near behaving" statements describe statements that have similar properties,
like data reuse/locality, parallelism, vectorization etc. that constitute their profile.

To achieve that, we first need to learn about the different properties. Then in a second
time, we need to found a way to measure/quantify them statement-wise. Then, in a third
time we need to find a way to compute, to \textit{rate}, the similarity of the different
properties between two statement. Then in the end, we will try to find smart algorithms
that produce the best aggregated set of statement, based on the rate computed.

%{The reasons behind the choice of this internship}
\section{The reasons behind the choice of this internship}
The first reason why I've asked and chosen this internship was that I wanted to try and learn
if I was capable, if I had what it takes to work in research, or if it was more than I could
chew.

The second was that I also wanted to put myself to the test and see if I really 
wanted to try to do a PhD or not : I wanted to know if I could be as interested in being
a researcher as I was interested in being a college teacher in Computer Science
(because it's hard in France to be the second without being the first one).

In other words,
I was expecting this internship to be a little bit like a trial by fire that would help me
to take a decision for my professional future.

\bigskip

In addition to these reasons, that were more personal and intimate than practical, there is
also the more practical reasons :
\begin{itemize}
    \item[] I wanted to learn more about optimization methods, both high and low level methods,
        so I know more about what can be done when trying to optimize by hand a code.
    \item[] I was expecting to learn some valuable skills :
        \begin{itemize}
            \item To be able to do the literature review about state of the art of
                a subject (here : code optimization techniques).
            \item To become more familiar with the reading, analyze and synthesis of
                scientific papers.
            \item To become more familiar with benchmarking, performance analyze and
                graphical representation of data.
            \item To become more familiar with the use of external libraries, project
                management and become more fluent in C and Python.
        \end{itemize}
\end{itemize}

\bigskip

Those were all the reasons that motivated my choice.


\section{General Informations}
%\section*{General Informations}
%\addtocontents{toc}{\protect\contentsline{section}{General Informations}{}}

    In this report, several things should be noted :
    \begin{itemize}
        \item Arrays used in code example are always in Row-major layout.
        \item Code example are either in C or pseudo-code.
    \end{itemize}

\section{Polyhedral Model}
The polyhedral model is (as explained in \cite{Bas'12}) a mathematical model that allows to represent program
part as union of polytopes and affine sets. Program parts that can fit the model
are called \textit{static control parts} or \textit{SCoPs} for short.
Scops are generally loop-based, and need to have affine loop bounds, conditions and
array subscripts to fit the polyhedral model.

The polyhedral model consist of three mathematical objects based on unions of relations :
\begin{itemize}
    \item The \textit{Iteration Domain}, that provide the set of all the iteration scanned
        %parcouru
        by a specific statement, using affine constraints. It's these affine constraints 
        that can be seen as a $\mathbb{Z}$-polytope.
    \item The \textit{Scattering Relations} (or \textit{Scheduling Relations},
        or even \textit{Mapping Relations}), that give us the ordering in which
        all the statement instances will be executed.
    \item The \textit{Access Relations} that models read/write accesses made by
        the different arrays and simple variables.
\end{itemize}

    \subsection{Iteration Domain}
        To understand Iteration Domain, it's necessary to first understand what a \textit{statement instance}
        is. A statement instance is one particular execution of a statement.
        When a statement is located inside a loop nest, it can be associated to the value
        of the outer-loop counters (called \textit{iterators}.
        The Iteration Domain will be the set of the iterators of all the statement
        instance.

        Because the outer loop bound can be affine or parameterized, it is generally impossible to
        know their actual value at compilation time. Even if we knew the value of the
        bounds, the number of iterators could be too enormous to count them all into sets.
        That's why the sets are expressed as a system of affine constraints.
        These affine constraints define a $\mathbb{Z}$-polyhedron, hence the name of the model.
        The relation used to represent the Iteration Domain is :
        \begin{center}
            $ \mathcal{D}_S(\vec{p}) = \left\{() \to \vec{\imath_S} \in \mathbb{Z}^{\dim(\vec{\imath_S})}
            \middle|
            \left[D_S\right]\begin{pmatrix}\vec{\imath} \\ \vec{p} \\ 1\end{pmatrix}
            \geq \vec{0}
            \right\}$
        \end{center}
        where $\vec{\imath}$ is the $\dim(\vec{\imath})$-dimensional iteration vector and
        $D_S \in \mathbb{Z}^{m_{\mathcal{D}_S} \times (\dim(\vec{\imath})+\dim(\vec{p})+1)}$
        is an integer matrix where $m_{\mathcal{D}_S}$ is the number of constraints.
        \\

\begin{lstlisting}[frame=single, language=C, caption={Simple code for polyhedral model example}, label={lst:polyhedral_example}]
        for(i=0 ; i<2*N - 1 ; i++)
S1:         A[i] = 42;
        for(i=0 ; i<N ; i++)
            for(j=0 ; j<N ; j++)
S2:             B[i][j] += 42;
\end{lstlisting}
        
        For example, in Listing~\ref{lst:polyhedral_example}, we have for
        S1 and S2 the iteration domain :
        \begin{itemize}
            \item[]$ \mathcal{D}_{S1}(N) = \left\{() \to \begin{bmatrix}i\end{bmatrix} \in \mathbb{Z} \middle|
            \left[\begin{array}{c:c:c}
                    1 & 0 & 0 \\
                    -1 & 2 & -2
            \end{array}\right]
            \left(\begin{array}{c}
                    i \\ \hdashline
                    N \\ \hdashline
                    1 
            \end{array}\right)
            \geq \vec{0}
            \right\} $,
        
            \item[]$ \mathcal{D}_{S2}(N) = \left\{() \to \begin{bmatrix}i\end{bmatrix} \in \mathbb{Z} \middle|
            \left[\begin{array}{c:c:c:c}
                    1 & 0 & 0 & 0\\
                    -1 & 0 & 1 & -1\\
                    0 & 1 & 0 & 0\\
                    0 & -1 & 1 & -1
            \end{array}\right]
            \left(\begin{array}{c}
                    i \\
                    j \\ \hdashline
                    N \\ \hdashline
                    1 
            \end{array}\right)
            \geq \vec{0}
            \right\} $.
        \end{itemize}

        In the case of \textit{if-then-else} enclosed statements, the iteration domain
        for the statement will be an union of disjoint polytope, each representing a part
        of the \textit{if-then-else} condition.

    \subsection{Scattering Relations}
        Defining the domain of iteration is great, but it doesn't give us any information
        on the order in which will be executed each statement instance.

        There is different type of ordering relations :
        \begin{description}
            \item[Scheduling] is about ordering the statement instances in time by
                giving them a logical date vector indicating when each will be executed.
                The data vector is generally read in \textit{lexicographically} order, for example
                $(1,2,3)$ is executed before $(1,2,4)$ but after $(1,1,4)$.
            \item[Allocation] or \textbf{placement} is about ordering statement instances
                in space by assigning to each statement instances a logical stamp
                indicating the processor number in which it will be executed.
                Statement instances not sharing the same logical stamp can be
                executed in parallel in different processor.
        \end{description}

        It is possible to express each ordering relations as separate relation,
        but it is also possible to aggregating them together : the result would be
        a relation between the iterators set and a set of ordering vector which
        would have some dimensions allocated to time ordering (scheduling) and
        some other to space ordering (allocation).

        For a statement $S$, the scattering relation can be written as follow :
        \begin{center}
            $\theta_S(\vec{p}) = \left\{\vec{\imath_S} \to \vec{t_S} \in \mathbb{Z}^{\dim(\vec{\imath_S})}\times\mathbb{Z}^{\dim(\vec{t_S})}
            \middle|
            \left[T_S\right]\begin{pmatrix}\vec{t_S}\\ \vec{\imath} \\ \vec{p} \\ 1\end{pmatrix}
            \geq \vec{0}
            \right\}$,
        \end{center}
        where $\vec{\imath}$ is the $\dim(\vec{\imath})$-dimensional iteration vector
        and $T_S \in \mathbb{Z}^{m_{\theta_S}\times(\dim(\vec{\imath_S})+\dim(\vec{t_S})+\dim(\vec{p})+1)}$
        is an integer matrix where $m_{\theta_S}$ is the number of constraints.

        For example, for Listing~\ref{lst:polyhedral_example}, we have for S1 and S2
        statement the original scattering of the program :
        \begin{itemize}
            \item[]
                $\theta_{S1}(N) = \left\{(i) \to \begin{pmatrix}t^{1}_{S1}\\t^{2}_{S1}\\t^{3}_{S1}\end{pmatrix} \in
                    \mathbb{Z}\times\mathbb{Z}^{3}
                    \middle|
                    \left[\begin{array}{ccc:c:c:c}
                            -1 & 0 & 0 & 0 & 0 & 0 \\
                            0 & -1 & 0 & 1 & 0 & 0 \\ 
                            0 & 0 & -1 & 0 & 0 & 0
                    \end{array}\right]
                    \left(\begin{array}{c}
                        t^{1}_{S1} \\
                        t^{2}_{S1} \\
                        t^{3}_{S1} \\ \hdashline
                        i \\ \hdashline
                        N \\ \hdashline
                        1
                    \end{array}\right)
                    = \vec{0}
                    \right\}$,\\
                    which can be simplify to $\theta_{S1}(N)(i)=\begin{pmatrix}0\\i\\0\end{pmatrix}$.
            \item[]
                $\theta_{S2}(N) = \left\{(i) \to \begin{pmatrix}t^{1}_{S2}\\t^{2}_{S2}\\t^{3}_{S2}\\t^{4}_{S2}\\t^{5}_{S2}\end{pmatrix} \in
                    \mathbb{Z}\times\mathbb{Z}^{3}
                    \middle|
                    \left[\begin{array}{ccccc:cc:c:c}
                            -1 & 0 & 0 & 0 & 0 & 0 & 0 & 0 & 1\\
                            0 & -1 & 0 & 0 & 0 & 1 & 0 & 0 & 0\\ 
                            0 & 0 & -1 & 0 & 0 & 0 & 0 & 0 & 0\\
                            0 & 0 & 0 & -1 & 0 & 0 & 1 & 0 & 0\\ 
                            0 & 0 & 0 & 0 & -1 & 0 & 0 & 0 & 0
                    \end{array}\right]
                    \left(\begin{array}{c}
                        t^{1}_{S2} \\
                        t^{2}_{S2} \\
                        t^{3}_{S2} \\
                        t^{4}_{S2} \\
                        t^{5}_{S2} \\ \hdashline
                        i \\ \hdashline
                        N \\ \hdashline
                        1
                    \end{array}\right)
                    = \vec{0}
                    \right\}$,\\
                    which can be simplify to $\theta_{S2}(N)\begin{pmatrix}i\\j\end{pmatrix}=\begin{pmatrix}1\\i\\0\\j\\0\end{pmatrix}$.
        \end{itemize}
        The first dimension for S1 is 0 and for S2 is 1 : it is to ensure that S2
        is always executed after S1.
        In our example, it doesn't matter because S1 and S2 doesn't have any data dependence
        whatsoever, and they could be run in parallel without problem, but for the sake
        of the example, we've imposed that S1 is executed before S2.

    \subsection{Access Relations}
        The \textit{Access Relations} models the memory access of the different
        array references in a statement. It map the statement instances to memory locations
        for every array reference in the statement. Non-array reference can be seen as
        0-dimensional array.

        The access relation can be written in the general form :
        \begin{center}
            $\mathcal{A}_{S,r}(\vec{p}) = 
            \left\{
                \vec{\imath_S} \to \vec{a}_{S,r} \in \mathbb{Z}^{\dim(\vec{\imath_S})}\times\mathbb{Z}^{\dim(\vec{a}_{S,r})}
                \middle|
                \left[A_{S,r}\right]\left(\begin{array}{c}\vec{a}_{S,r}\\\vec{\imath_S}\\\vec{p}\\1\end{array}\right)
                \geq \vec{0}
            \right\}$
        \end{center}
        where 
        \begin{itemize}
            \item $r$ is the number of the reference in the statement (every references are
                numbered, from left to right generally),
            \item $\vec{\imath}$ is the $\dim(\vec{\imath})$-dimensional iteration vector
            \item $T_S \in \mathbb{Z}^{m_{\theta_S}\times(\dim(\vec{\imath_S})+\dim(\vec{t_S})+\dim(\vec{p})+1)}$
                is an integer matrix where $m_{\theta_S}$ is the number of constraints.
        \end{itemize}

        The read or write nature of the memory access is not expressed in this form,
        and need to be given (for example, to compute data dependences).

        %TODO Expliquer les outputs dimensions (si il y en as dans le modèle théorique).

        For example, for Listing~\ref{lst:polyhedral_example}, we have for S1 and S2 the
        access relation :
        \begin{itemize}
            \item[]
                $
                \mathcal{A}_{S1,1}(N) = 
                \left\{
                \begin{pmatrix}i\end{pmatrix} \to \left(a_{S1,1}\right) \in \mathbb{Z}\times\mathbb{Z}
                    \middle|
                    \left[\begin{array}{c:c:c:c}-1 & 1 & 0 & 0 \end{array}\right]
                    \left(\begin{array}{c}a_{S1,1}\\\hdashline i\\\hdashline N\\\hdashline 1\end{array}\right)
                    = \vec{0}
                \right\}
                $
            
            \item[]
                $
                \mathcal{A}_{S2,1}(N) = 
                \left\{
                \begin{pmatrix}i\end{pmatrix} \to \left(a_{S2,1}\right) \in \mathbb{Z}\times\mathbb{Z}
                    \middle|
                    \left[\begin{array}{c:c:c:c:c}-1 & 1 & 0 & 0 & 0 \end{array}\right]
                    \left[\begin{array}{c:c:c:c:c}-1 & 0 & 1 & 0 & 0 \end{array}\right]
                    \left(\begin{array}{c}a_{S2,1}\\\hdashline i\\\hdashline j\\\hdashline N\\\hdashline 1\end{array}\right)
                    = \vec{0}
                \right\}
                $
        \end{itemize}

        Because S1 and S2 only have one reference, there's only one access relation for
        each of them.

    \subsection{Dependence Relations}
    \label{sec:dependence_relations}
        Data Dependences are necessary when, for example, we have to identify if a statement
        can be parallelized or vectorized (simdized). They are also needed when we try to
        transform the scattering of a statement : doing that could violate some dependences,
        and the Dependence Relations (with Violation Relations) can be used to detect such
        violations.

        Fortunately, Data Dependence can be easily represented in the Polyhedral Model as
        \textbf{Dependence Relation}.

        A Dependence Relation represent the fact that some ordered statement iterations
        access the same memory location. The general form of the Dependence Relation of
        a dependence between a \textit{source} statement $S$ and a \textit{target}
        statement $T$ at a depth $d$ is the following (explained after) :

        \begin{center}
        \scalebox{0.85}{
            ${
            \delta_{S,r_S \xrightarrow{d} T,r_T}(\vec{p}) =
                \left\{
                \begin{pmatrix}\vec{\imath}_S \\ \vec{a}_{S,r_S} \end{pmatrix}
                \to
                \begin{pmatrix}\vec{\imath}_T \\ \vec{a}_{T,r_T} \end{pmatrix}
            \middle|
                \left[\begin{array}{c:c|c:c|c:c}
                    D^{\vec{\imath}_S}_{S} & 0 & 0 & 0 & D^{\vec{p}_S}_{S} & D^{c}_{S} \\ \hdashline
                    0 & 0 & D^{\vec{\imath}_T}_{T} & 0 & D^{\vec{p}_T}_{T} & D^{c}_{T} \\ \hline
                    A^{\vec{\imath}_S}_{S,r_S} & A^{\vec{a}_{S,r_S}}_{S,r_S} & 0 & 0 & D^{\vec{p}}_{S,r_S} & D^{c}_{S,r_S} \\ \hdashline
                    0 & 0 & A^{\vec{\imath}_T}_{T,r_T} & A^{\vec{a}_{T,r_T}}_{T,r_T} & D^{\vec{p}}_{T,r_T} & D^{c}_{T,r_T} \\ \hdashline
                    0 & I & 0 & -I & 0 & 0 \\ \hline
                    I^{1..d-1,\bullet} & 0 & -I^{1..d-1,\bullet} & 0 & 0 & 0 \\ \hdashline
                        I^{d,\bullet} & 0 & -I^{d,\bullet} & 0 & 0 & 0\text{ or }-1 \\
                \end{array}\right]
                %\quad
                \begin{pmatrix}{c}
                    \vec{\imath}_S \\ \hdashline
                    \vec{a}_{S,r_S} \\ \hline
                    \vec{\imath}_T \\ \hdashline
                    \vec{a}_{T,r_T} \\ \hline
                    \vec{p} \\ \hdashline
                    1 \\
                \end{pmatrix}
                \quad
                \begin{matrix}
                    \geq \\ \hdashline
                    \geq \\ \hline
                    \geq \\ \hdashline
                    \geq \\ \hdashline
                    = \\ \hline
                    = \\ \hdashline
                    \geq \\
                \end{matrix}
                \vec{0}
                \right\}
            }$
        }
        \end{center}
        If, for the statement $S$ and $T$ and with the depth $d$, this relation is not empty
        then there is a dependence between $S$ and $T$.
        
        This matrix seems complicated but we are just piling up more and more constraints
        on a linear system. Theses constraints correspond to three sub-relations :
        \begin{itemize}
            \item First, there is the \textbf{Existence Condition} of the instances.
                It corresponds to the line 1-2 of the matrix.

                The \textit{Existence Conition} determine, like in the \textit{Domain Relation},
                the set of possible iterations for the 2 statements (a dependence couldn't
                possibly exist between out-of-\textit{Domain} iterations).

                It's just the \textit{Domain Matrix} of the 2 statements split vertically
                in 3 and placed correctly in the \textit{Dependence Matrix}.
            \item Second, there is the \textbf{Conflit Condition} of the memory location.
                It corresponds to the line 3-4-5 of the matrix.

                The \textit{Conflict Condition} determines the equality of the 2 different 
                memory accesses causing the dependence : if there is a dependence,
                it means that for some iterations, there is 2 memory accesses that 
                access the same memory location, meaning that for some iterations,
                the 2 memory accesses are equal.

                So we add to the system the \textit{Access Matrix} of the memory accesses of the 2
                statements causing the dependence and then a series a constraints
                that bind the two access into equality (Identity matrix for the first
                \textit{Access Matrix} , and a negative Identity matrix for the second).
            \item Third, there is the \textbf{Causality Condition} of the instances.
                It corresponds to the line 6-7 of the matrix.

                The \textit{Causality Condition} make sure that the iterations of the
                \textit{source} statement are executed before the ones of the \textit{target} statements
                involved in the dependence.

                We define what is called the \textit{dependence depth} $d$. There is
                as many possible $d$ as there is of common loops between the 2 statements.
                If for a value of $d$, the linear system represented by the \textit{Dependence Matrix}
                has a solution, it means that for the loop of depth $d$ carry the dependence.
                If multiple value of $d$ produce solution to the linear system (so there
                is multiple \textit{Dependence Relation} for the same statements and
                memory location/accesses), it means that multiple loops carry the dependence.

                To add the notion of \textit{Causality} we add to the system a series of
                constraint binding the corresponding \textit{source} and \textit{target}
                iterators of the loops having a depth lesser than $d$
                (with a Identity and negative Identity).
                For the \textit{source} and \textit{target} iterator of the loop having
                the depth $d$, instead of an equality, we bind them with inequality :
                the \textit{source} iterator lesser or equal ($\leq$) to the \textit{target} iterator.

                It's also possible to force a strict inequality ($<$) by setting the cell
                in the last line and column of the matrix to $-1$, instead of $0$.
        \end{itemize}

        \medskip
        
        To illustrate the \textit{Dependence Matrix}, we would like to introduce a piece of code :
\begin{lstlisting}[frame=single, language=C, caption={Simple code for Dependence Relation example}, label={lst:dependence_example}]
for(i=1 ; i<N ; i++)
{
        A[i] = 42;
        B[i] = A[i-1]
}
\end{lstlisting}
        
        For example, in the Listing~\ref{lst:dependence_example}, we can see a Read-after-Write
        dependence : when $(i=1)$ we write in $A[1]$, and when $(i=2)$ (the next iteration)
        we read in $A[1]$ (and write in $A[2]$ but the location is different so that's not important here).

        For this dependence, we would have the following \textit{Dependence Matrix} :
\begin{center}
$
        \begin{array}{c:c|c:c|c:c}
        i &    [0] & i' &   [0]' &  N &  1 \\ \hline \hline
        1 &    0 &   0 &    0 &   0 &  -1 \\
        -1 &    0 &   0 &    0 &   1 &  -1 \\
        0 &    0 &   0 &    0 &   1 &  -2 \\ \hdashline
        0 &    0 &   1 &    0 &   0 &  -1 \\
        0 &    0 &  -1 &    0 &   1 &  -1 \\
        0 &    0 &   0 &    0 &   1 &  -2 \\
        1 &   -1 &   0 &    0 &   0 &   0 \\ \hline
        0 &    0 &  -1 &    1 &   0 &   1 \\ \hdashline
        0 &   -1 &   0 &    1 &   0 &   0 \\ \hline
        -1 &    0 &   1 &    0 &   0 &   0 \\
        \end{array}
        \qquad
        \begin{matrix}
                \\ \hline \hline
        {i-1 >= 0} \\
        {-i+N-1 >= 0} \\
        {N-2 >= 0} \\ \hdashline
        {i'-1 >= 0} \\
        {-i'+N-1 >= 0} \\
        {N-2 >= 0}  \\
        {i-[0] == 0}\\ \hline
        {-i'+[0]'+1 == 0} \\ \hdashline
        {[0] - [0]' == 0}  \\ \hline
        {-i+i' >= 0} \\
        \end{matrix}
$
\end{center}
    with :
    \begin{itemize}
        \item $[0]$ and $[0]'$ the first (and only) dimension of respectively \verb'A[i]' and
            \verb'A[i-1]'.
        \item $i$ and $i'$ the iterators of respectively \verb'A[i]' and \verb'A[i-1]'
            (here $i$ and $i'$ are the same, but iterators are always considered different).
        \item A dependence depth $d = 1$.
    \end{itemize}
    
    If we pass this matrix in any linear solving software (like PIP~\cite{Fea88}) and try to solve it,
    we would get solutions, meaning that the dependence exists.

    \bigskip

    That's really one of the most powerful side of the Polyhedral Model, its ability to
    compute the different dependences by building linear systems : if there is solution,
    there is dependence, and if not, there is not dependence. It's powerful because
    the techniques to build and solve linear system are known and mastered, so the basis
    of the model is really strong and stable. 


    \subsection{OpenScop Framework}
        OpenScop~\cite{openscop} is an open specification that define a file format
        and a set of data structures to represent a static control part (SCoP).

        OpenScop format is intended to be a stable bridge between polyhedral tools
        and optimizers allowing interchangeable tools and flexible toolchains.

        A library named \textit{osl} is provided to read/write OpenScop files
        and manipulate the different data structures the specification define.
        The file format as well as the library has been designed to accept extensions.

        Substrate, the tool performing our analyze and optimization pass uses OpenScop
        as input and output format.

\section{Statement Profiling}
\label{sec:statement_profiling}
    \subsection{Data Reuse/Locality}
        In this section, we will explain the basics behind data reuse/locality and
        then talk about the different metrics and methods found in the literature used to evaluate
        and quantifying data reuse and why they where selected or not for the statements profile.

        %XXX
        %Le problème étant que dans notre cas, seul la quantification de reuse nous intéresse
        %la locality étant lié au scattering, il nous intéresse moins.

        \subsubsection{Difference between Reuse and Locality}
            It's important to differentiate \textit{reuse} from \textit{locality}.
            We say that there is \textit{reuse} for a data when this data is
            access multiple time during the execution of a loop nest, during
            the same iteration or not. It can be from different references or different statements.
            
            When there is reuse, it's possible (but not guaranteed) to have
            \textit{locality}. We say that there is \textit{locality} when the
            reused data remains in memory hierarchy level targeted 
            (eg : the CPU cache in our case) between 2 reuses.
            For that to be the case, the amount of other data accessed between the
            2 reuses must not exceed the cache size, or else the reused data will be
            evicted from the cache, causing a cache miss and forcing the cache
            to fetch the data from the RAM (and RAM have higher latency compared to CPU cache).


            Thus, we can say that the \textit{reuse} is linked to the way indexing
            functions (for array references) are written, whereas \textit{locality} is
            inherent to the iterations order/the way the loops are written.

        \subsubsection{Types of Reuse}
            
\begin{lstlisting}[frame=single, language=C, caption=Reuse example, label={lst:reuse_example}]
for(i=0; i<N ; i++)
    for(j=0; j<N ; j++)
    {
        A[i][j] = B[i] + B[i+1];
        C[i][j] = B[i+2] * 2;
    }
\end{lstlisting}

            \paragraph{Self-Temporal}
                Self-Temporal reuse occurs when the same reference access the same
                data location.
                
                For example, in Listing \ref{lst:reuse_example}, the reference \verb'B[i]' carry Self-Temporal
                reuse :\verb'B[0]',\verb'B[1]',\verb'B[2]' ... \verb'B[N-1]' are accessed
                \textit{j} times by the same reference \verb'B[i]'.
                And for the same reason, \verb'B[i+1]' and \verb'B[i+2]' also carry Self-Temporal reuse.

            \paragraph{Self-Spatial}
                Self-Spatial reuse occurs when the same reference access the same
                line in an array (when in row-major). Especially, the space between
                two accessed data locations should not be greater than a cache line size,
                or else there won't be any spatial locality.

                For example, in Listing \ref{lst:reuse_example}, the reference \verb'A[i][j]' carry
                Self-Spatial reuse.

                Self-Temporal reuse can be considered as a special case of
                Self-Spatial reuse : accessing the same data location mean that
                the same array line is also accessed.
            \paragraph{Group-Temporal}
                Group-Temporal reuse occurs when different references access the same
                data location. It can occur inside a same statement or across different
                statements.

                For example, in Listing \ref{lst:reuse_example}, the reference
                \verb'B[i]', \verb'B[i+1]' and \verb'B[i+2]' carry Group-Temporal :
                the data accessed by \verb'B[i+2]' when $i=M$ will be access by
                \verb'B[i+1]' when $i=M+1$ and by \verb'B[i]' when $i=M+2$.

            \paragraph{Group-Spatial}
                Group-Spatial reuse occurs when different references access the same
                line in an array (when in row-major). Again, the space between
                two accessed data location should not be greater than a cache line size,
                or else there won't be any spatial locality.
                
                For example, in Listing \ref{lst:reuse_example}, the reference
                \verb'B[i]', \verb'B[i+1]' and \verb'B[i+2]' carry Group-Spatial :
                they all access to different data location, but on the same line.\\
                \\

                So as we can see, a single reference can carry multiple type of reuse.
                A statement being generally composed of multiple references, it 
                can be hard to automatically favor some statements instead of others.
        \subsubsection{Reuse Vector Space}
            The notion of Reuse Vector Space come from Wolf et. al.\cite{Wolf'91} and describe
            the vector subspace of the Iteration vector space where the reuse of
            a reference can be taken advantage to induce locality.
            There is as many reuse vector space as there is type of reuse.

\begin{lstlisting}[frame=single, language=C, caption=Simple example, label={lst:simple_example}]
for(i=0; i<N ; i++)
    for(j=0; j<N ; j++)
    {
        f(A[i], A[j]);
    }
\end{lstlisting}

            For example, in Listing \ref{lst:simple_example}, with the Iteration
            space being spanned by $\{(1,0),(0,1)\}$, the self-spatial reuse vector space of
            \verb'A[i]' is $\{(1,0)\}$ (ie the {\it i} loop), and the one of \verb'A[j]' is $\{(0,1)\}$
            (ie the {\it j} loop).

            Wolf et. al. use the concept of uniformly generated references proposed
            by Gannon et. al.\cite{Gannon:1988:SCL:50454.50460} for estimating reference
            windows :
            \begin{defn}
            \label{sec:uniformly_generated}
                Let $n$ be the depth of a loop nest, and $d$ be the dimensionality of
                an array $A$. Two references $A[\vec{f}(\vec{\imath})]$ and
                $A[\vec{g}(\vec{\imath})]$, where $\vec{f}$ and $\vec{g}$ are indexing
                functions $Z^{n} \rightarrow Z^{n}$, are called \textbf{uniformly generated} if :
                \begin{center}
                    $\vec{f}(\vec{\imath}) = H\vec{\imath}+\vec{c}_f$ and $\vec{g}(\vec{\imath}) = H\vec{\imath}+\vec{c}_g$
                \end{center}
                where $H$ is a linear transformation and $\vec{c}_f$ and $\vec{c}_g$ are constant vectors.
            \end{defn}
            
            We can partition references into equivalence class, called \textit{uniformly generated sets},
            when the references operate on the same array, and have the same $H$.
            We can see $H$ as a part of one of the access function (in the Polyhedral Model representation) of this statement :
            it's the access function matrix without the part handling the constant vector.

            For example in Listing \ref{lst:reuse_example}, $B[i]$,$B[i+1]$ and $B[i+2]$ can be each written as
            \begin{center}
                $
                \begin{bmatrix}
                    1 & 0
                \end{bmatrix}
                \begin{bmatrix}
                    i\\
                    j
                \end{bmatrix}
                +
                \begin{bmatrix}
                    0
                \end{bmatrix}
                $,\\
                $
                \begin{bmatrix}
                    1 & 0
                \end{bmatrix}
                \begin{bmatrix}
                    i\\
                    j
                \end{bmatrix}
                +
                \begin{bmatrix}
                    1
                \end{bmatrix}
                $,\\
                $
                \begin{bmatrix}
                    1 & 0
                \end{bmatrix}
                \begin{bmatrix}
                    i\\
                    j
                \end{bmatrix}
                +
                \begin{bmatrix}
                    2
                \end{bmatrix}
                $.
            \end{center}
            As we can see, all these reference share the same $H=\begin{bmatrix}1 & 0\end{bmatrix}$,
            so they all belong to the same class.\\
            Of course, $B[i]$,$B[i+1]$ and $B[i+2]$ belong to different statement, and in
            our pass, a uniformly generated set would be created for each statement.

            The problem with that method, is that it would put the references
            $C[i][j],C[i][j+1],C[i-1][j]$ in the same class : they all share the
            same $H$, but they don't have the same reuse space at all.
            For example : $C[i][j],C[i-1][j]$ exhibit group-spatial reuse in the $i$ dimension,
            whereas $C[i][j],C[i][j+1]$ exhibit group-spatial reuse in the $j$ dimension. So they
            should not be group together in the same class.
            \medskip

            So the definition of the \textit{uniformly generated set} is not restrictive enough,
            and in out case, we propose an addition to it : for two references $A$ and
            $B$ (as expressed in Definition~\ref{sec:uniformly_generated}) to be in the same class :
            \begin{itemize}
                \item They need to have the same $H$.
                \item with $\vec{c}=\vec{c_g} - \vec{c_f}$, $\vec{c}$ need to have $0$
                    for all the elements of the vector \textbf{except one}.
            \end{itemize}

            %TODO clarifier les choses, et peut être au lieu de mentionner les méthodes
            %de Wolf, les expliquer içi.

            The reuse spaces would then be computed for each type of reuse
            with the methods described in~\cite{Wolf'91} : statements where
            array references share the same reuse space should be aggregated (if
            it doesn't violate any data dependence).



        \subsubsection{The Simple Algorithm}
            The \textit{Simple Algorithm}~\cite{Kennedy94maximizingloop} is a
            optimization algorithm that improve data locality via Loop Fusion.
            The idea is to fusion vertices of a directed graph when these vertices
            are connected by non-fusion-preventing arcs.
            In this graph, the vertices represent the different loop nests of a program part,
            and the arcs represent data dependences between statements of different loop nests.

            If the dependence is fusion-preventing~\cite{Bacon:1994:CTH:197405.197406} or if the loop headers are incompatible,
            the arc representing the dependence will be marked fusion-preventing.

            The algorithm is a modification of the greedy algorithm that partition
            the nodes of the graph without sequentializing a parallel node/loop, and
            is the following (directly taken from~\cite{Kennedy94maximizingloop}:
            \medskip

            ``  In the greedy partition graph, if there exists two partitions $g_1$
                and $g_2$ with a directed edge $(g_1,g_2)$, then no node in $g_2$
                may be legally placed in $g_1$ or the greedy algorithm would have put it
                there. However, it may be safe and profitable to move nodes in $g_1$
                down into $g_2$.

                We determine if it is safe and profitable to move $n \in g_l$ down
                into another partition $g_h$ as follows.
                \begin{description}
                    \item [Safety.] A node $n \in g_l$ may move to $g_h$ \textit{iff} it
                        has no successors in $g_l$ and there is no fusion-preventing
                        edge $(n,r)$ such that $r \in g_h$ or $r \in g_k$ where
                        $g_k$ must precede $g_h$.
                    \item [Profitability.] Compute a sum of edges $(m,n), m \in g_l$
                        and a sum for each $g_h$ of edges $(n,p)$ such that $p \in g_h$
                        and $n$ may be safely moved into $g_h$. Pick the partition
                        $g$ with the biggest sum. If $g \neq g_l$, move $n$ down into $g$.''
                \end{description}
            \bigskip
            
            This algorithm doesn't give us any metric or way to evaluate a statement
            reuse, so it doesn't help use for the reuse evaluation side, but the idea
            of a dependence graph based greedy algorithm is interesting : with statement
            instead of loop nest being the nodes, this kind of greedy algorithm 
            could fusion statements that are inside different loop nest in the SCoP.
            It means we would also need to redefine the \textbf{Safety} and \textbf{Profitability}
            notions of the algorithm.

        \subsubsection{Data and Loop Transformations Algorithm}
            An algorithm that transforms a loop nest to improve cache locality
            by a combination of loop and data transformations is presented in~\cite{Kandemir99improvingcache}.

            There's not much to say about it. The goal of the algorithm is to
            compute a transformation matrix and array layout (and the explanation is a little
            vague), so it isn't applicable in our case and it doesn't give us any metric or way to evaluate the reuse.
        
        \subsubsection{LoopCost Algorithms}
            The LoopCost algorithm~\cite{McKinley:1996:IDL:233561.233564} provide
            us with a cost model that allow to guide some loop transformation like
            loop permutation, loop reversal, loop fusion, loop distribution,
            by comparing the cost with and without the transformation.
            
            The LoopCost algorithm work like this : 
            Each loop $l$ in a loop nest is considered as a candidate for the innermost
            position in the nest. For each loop $l$, LoopCost compute and sum the number
            of cache line access by a group of references (the groups are formed
            using the RefGroup algorithm) when $l$ is placed as the innermost loop.

            As a result, we have a cost for each loop when they are considered as innermost.
            This cost enable to possibility to compare with hard numbers
            a loop nest before and after some loop transformations, which allow to discard
            or apply the transformations if the new cost is worse or better than the
            original cost.
            \bigskip

            Because the LoopCost is loop transformation oriented and because it doesn't
            use the number of references in a reference group for the computation
            of the cost, it is not really applicable to our situation.

            Regarding the RefGroup algorithm that partition the array references
            in group, the criteria used are like the ones for \textit{uniformly generated set}
            but a little more restrictive, and we don't need more restrictive, so we won't
            be able to use them.
            
            
        \subsubsection{Automatic Polyhedral Parallelization and Locality Optimization Algorithm}
            Let $S_i$ and $S_j$ two statement of a program with a data dependence $e$
            from $S_i$ to $S_j$.\\
            Let $\vec{s}$ and $\vec{t}$ the iterators of respectively $S_i$ and $S_j$.\\
            Let $\theta_{S_i}$ and $\theta_{S_j}$ the scattering function of respectively $S_i$
            and $S_j$.\\
            Consider the following affine $\delta_e$ :
            \begin{center}
                $\delta_e\left(\vec{s},\vec{t}\right) = \theta_{S_j}\left(\vec{t}\right) - \theta_{S_i}\left(\vec{s}\right)$
            \end{center}

            According to the paper~\cite{Bondhugula:2008:PAP:1379022.1375595},
            the affine form $\delta_e\left(\vec{s},\vec{t}\right)$ is very significant
            because it gives us ``the number of hyperplanes the dependence $e$ traverses
            along the hyperplane normal $\theta$. If $\theta$ is used as a space loop to generate tiles
            for parallelization, this function is a factor in the communication
            volume. On the other hand, if $\theta$ is used as a sequential loop, it
            gives us a measure of the reuse distance.''

            If there is a upper bound to $\delta_e\left(\vec{s},\vec{t}\right)$,
            minimizing it (or even nullify it), would be the same as minimizing (or nullify)
            the number of hyperplanes that would be communicated as a result of
            the dependence at the tile boundaries.

            The problem is that minimizing $\delta_e\left(\vec{s},\vec{t}\right)$
            can be really difficult because it ends up in an objective non-linear
            in loop variables and hyperplane coefficients.

            We can avoid this problem if we use a bounding function $v(\vec{p}) = u.\vec{p} + w$ such that :
            \begin{center}
                $\delta_e\left(\vec{s},\vec{t}\right) \leq v(\vec{p})$
            \end{center}
            i.e.,
            \begin{center}
                $v(\vec{p}) - \delta_e\left(\vec{s},\vec{t}\right) \geq 0$
            \end{center}

            With this last inequality, we can apply the Farkas Lemma :
            \begin{center}
                $v(\vec{p}) - \delta_e\left(\vec{s},\vec{t}\right) \equiv \lambda_{e0}
                + \sum\limits_{k=1}^{m_e}{\lambda_{ek}\mathcal{P}_{e}^k},\qquad \lambda_{ek} \geq 0$
            \end{center}
            where $P_e^k$ is a face of $P_e$.

            From this new form, we can get a linear system (formed by equalities and
            inequalities), which can then be handled by PIP software to find
            minimal solution to it.

            This algorithm doesn't give us any metric or way to evaluate reuse, but instead
            give us a way to find automatically legal transformations, so we can't use it.

%        \subsubsection{Ehrhart Polynomial Algorithm}
%            TODO

        \subsubsection{The chosen model for Reuse Profile}
            %TODO

    \subsection{Parallelism}
        We say that a statement can be parallelized for a certain loop when it is legal to
        execute different \textit{instances} of the statement (so with different iterators)
        on multiple cores at the same time, without violating any dependences. The
        goal (not always possible) of executing multiple instances at the same time is to
        speed up the overall execution of the program.

        Unlike \textit{Data Reuse} where we had to find a good way to detect and quantify it
        because there could be more or less \textit{reuse} in a statement, the \textit{Parallelism}
        of a statement doesn't need to be "quantify" : a statement can be either parallelized
        for a certain loop or it cannot : no more, no less.

        So in the end, the \textit{Parallelism Profile} of a statement is just the list
        of loop than can be parallelized safely. And to determine which loop can be parallelized
        safely, we just need to analyze the \glspl{self-dependence} of the statement.
        We only analyze \gls{self-dependence} because we analyze the statement alone, 
        like if there is no other statement in the scop, so the only possible dependences
        are the one from the statement to itself.
        Dependences in the Polyhedral Model are explained in the Section~\ref{sec:dependence_relations}.
       
        \bigskip

        Because we analyze the \glspl{self-dependence} of a statement, the only
        dependence we will find to analyze are Read-after-Write or Write-after-Read dependences :
        \begin{itemize}
            \item Read-after-Read dependences doesn't prevent parallelization because there is no
                change made at the memory location in question, so we don't even look at them.
        
            \item Write-after-Write dependences doesn't concern us here because there is
                only one \textit{write} to the memory per statement which means that
                the \textit{source} and \textit{target} statement of Write-after-Write
                dependences are always different, which is not out case here.
        \end{itemize}
        
        So once we've computed all the dependence possible (for all the depth $d$ possible),
        we need, for all the iterators/loops of the statement, to check if one of the dependences
        is loop-carried :
        \begin{itemize}
            \item A statement can be parallelized for a certain loop if none of its \glspl{self-dependence}
                are loop-carried.
            \item In the contrary if any \gls{self-dependence} of a statement is loop-carried for an certain loop,
                the statement cannot be parallelized for this loop.
        \end{itemize}
        Moreover, if a statement doesn't have \textit{any} dependence, it means that the
        statement can be parallelized for all its loops.

        \bigskip

        To determine if a \textit{Dependence Relation} is loop-carried or not for a certain
        loop $l$, we just have to add to the linear system represented by the
        \textit{Dependence Matrix} the following constraints :
        \begin{itemize}
            \item A equality constraint between each of the $(l-1)$ first iterators
                of the \textit{source} and \textit{target} statements.
            \item A \underline{strict} inequality between the $l^{\text{th}}$ iterator of
                the \textit{source} and \textit{target} statements.
        \end{itemize}
        These additional constraints have the following meaning : if dependences
        still exists between \textit{instances} of the \textit{source} and \textit{target}
        statement in the $l^\text{th}$ iteration dimension
        \begin{itemize}
            \item even when all of their \textit{other} iterators are equal (the "equal" constraints),
            \item even when the \textit{source instance} and \textit{target instance} of the dependences
                in the $l^\text{th}$ must be strictly different (the "strict inequality" constraint),
        \end{itemize}
        then the dependence is loop-carried for the loop $l$.

        Once the constraints have been added to the system, we just have to give it to a
        solver like PIP : if there is a solution, the dependence is loop-carried for $l$,
        else it is not.
        
    \subsection{Vectorization}
        \subsubsection{Principles}
        Vectorization is a special case of Parallelization where we try unroll a little
        the loop and transform scalar operations, that compute only one single pair of
        operand, to vector operations (SIMD operation : Single Instruction, Multiple Data),
        that compute multiple pair of operand at a time, so that the loop can run a little
        faster and more efficiently.

        One of the condition for a statement to be vectorized is that the memory accesses
        that appear in it should have the same alignment in memory : the address of all the
        accesses of modulo the vector register size should be the same for all the accesses
        of the statement.
        But techniques exist to transform non-aligned accesses to aligned accesses, like
        in this paper~\cite{Eichenberger:2004:VSA:996893.996853}, so we chose to ignore this
        condition, by considering that compilers are now capable of handling this problem.

        Because Vectorization is a special case of Parallelization, the way to detect it
        is a lot like the way to detect parallelization : it involve the analyze of
        the \glspl{self-dependence} of a statement. But unlike Parallelization, here we are
        only interested in a certain number of loops. Those loops are the ones whose iterators
        are used either in :
        \begin{itemize}
            \item the last output dimension of the accesses of the statements when in row-major,
            \item the first output dimension of the accesses of the statements when in column-major.
        \end{itemize}
        These loops also need to have a unit stride (increment of 1).

        We will call the loops validating these 2 conditions \textbf{candidate loops}.
        The other loops, the ones that don't validate the 2 conditions, are ignored for
        the rest of the analyze (even though they could have been parallel). This is
        because when we vectorize a statement, the accesses of consecutive instances have
        to access consecutive memory location which can only be the case when the iterator
        of the loop is used in the last/first output dimension and when it have a unit stride.
        To have a real unit stride, these loop also need to be the innermost loop of the loop nest
        surrounding the statement.
        
        But we consider that it is not necessary for a loop to be the innermost to be
        in the \textit{Vectorization Profile}, because to be in the profile
        we only want to know which loop \textbf{could} allow vectorization
        for the statement \textbf{if} this loop was the innermost. The loop may not be
        the innermost at the moment, but the \textit{Vectorization Profile} would tell
        us that if we find a way to make it the innermost, then we would unlock vectorization
        for this statement and for this loop.

        \bigskip

        We then need to analyze the \glspl{self-dependence} of the statement :
        \begin{itemize}
            \item If there is no dependence at all, then we can add all the 
                \textit{candidate loops} to the \textit{Vectorization Profile}.
            \item If there is dependences, we only look at the dependences that
                are carried by one or more of the \textit{candidate loops}.\\
                For all the remaining dependences $D$ carried by the loop $l$,
                if $D$ is a Write-after-Write or Read-after-Write dependence, then $l$
                is removed from the \textit{candidate loops}.
                The other kinds of dependence are Write-after-Read Read-after-Read 
                which doesn't cause any problems for the vectorization, so we can ignore
                those kinds of dependences.
        \end{itemize}
        The \textit{candidate loops} that are left are then added to the profile because they
        don't cause any threats to the vectorization.

        \bigskip

        The \textit{Vectorization Profile} is in the end just the list of loops that
        \textbf{if} they are/become the innermost loop, we can then vectorize the statement.

        \subsubsection{Limitations}
        For the \textit{Vectorization Profile}, we have chosen to only analyse the statements
        as if they were the only statement of the loop nest : that's why we only talk
        about dependences that are from and to the statement.

        This approach simplify the problem, but add a limitation to the profile :
        a statement could be vectorized if it is considered "alone" but can in fact not
        be vectorized because it has some dependences with some other statements of the loop
        nest.

        For example, in the following code :
\begin{lstlisting}[frame=single, language=C, caption={Vectorization Profile limitation example}, label={lst:vectorization_limitation}]
for(i=1 ; i<N ; i++) {
    A[i-1]=B[i]+1; // S1
    C[i]=A[i]*2; // S2
}
\end{lstlisting}
    S1 doesn't have any \gls{self-dependence}, so the $i$ loop would be in its
    profile. But there is a Write-after-Read dependence between S1 and S2 : this dependence
    make the vectorization of S1 illegal.

    So when using the \textit{Vectorization Profile} during an optimization, we need to keep
    checking for inter-statement dependences when trying to vectorize statements.

    %XXX if the size of the vector register is smaller than minimum of all the access
    %A[i-offset], then we can still vectorize, but it's a borderline case.

    \subsection{Tiling Hyperplane}
        %TODO
    %\subsection{Register Pressure}

\section{SuBStrAte : Aggregating "Near Behaving" Statements Using Profiling}
    Substrate is the result and the implementation of the work and the research I've done
    about how to profile the different "attribute" of a statement.\\
    Substrate is a program that will profile each statement of a \textit{scop} for
    each each of its "attribute" with the analyze pass.

    It will then try to aggregate
    statements into what I call \textbf{meta-statement}. Meta-statements are simply
    aggregated statements seen as one statement (in the way the polyhedral see statements)
    : it has the same \textit{Domain} and \textit{Scattering} than all its statements,
    all the memory accesses of all the statements are the memory accesses of the \textit{meta-statement}.

    To aggregate statements into meta-statements, we need to rate the similarity between two
    statements (or meta-statements) : putting a numerical value to the similarity.
    If the similarity is high, the rate is high, and if the rate is high, we need to
    aggregate the statements.

    Si I had to find a way to rate this similarity : transforming the comparison between
    2 profile into a numerical value.

    \subsection{Statement Profiling}
    The analyze pass is the direct implementation of all the research done in the section~\ref{sec:statement_profiling}:
    \begin{itemize}
        \item The data reuse profile is built by grouping the
            different memory accesses into array reference group, then uniformly generated set group,
            then group with the same reuse direction. The harder part here wasn't the
            implementation but the research part.

        \item The Parallelism profile is built by listing all the loop index that could be
            parallelized for the statement. It is done by using the loop carried algorithm
            of the polyhedral model : if for a loop index, not a single one dependence is
            loop carried by it, then it can be parallelized safely.
            The catch in our case was that the implementation of this algorithm was in
            \textbf{CAnDL}, another library/program of the \textit{periscop} suite (a suite of
            library/program using the OpenScop format). \textit{CAnDL} can compute
            a lot of things regarding the \textit{dependence} in the \textit{Polyhedral Model},
            but its implementation of the loop carried algorithm wasn't working due to old
            changes in \textit{OpenScop} that were never correctly/fully implemented for this
            function/algorithm.
            
            So I had to fix the implementation without breaking anything.
            Some parts of the code weren't making any sens, slowing me down because I had
            to figure out what I could safely remove, what I couldn't remove, and what I
            had to fix.

            In the end, the function is now working, and as a side effect of the time I've taken
            to fix it, I've added to \textit{CAnDL} a function that compute if a dependence
            is loop independent or not.
        \item The Vectorization profile building algorithm
            , like I explained in the section~\ref{sec:statement_profiling}, is a lot like
            the Parallelism profile building algorithm, but with some additional checks.
            In the end, the Vectorization profile is simply the list of loop index for which
            the statement \textit{could be} vectorized if they were the innermost loop of the loop nest.\\

            And because the fix of the loop carried function of \textit{CAnDL} was done during
            the implementation of the Parallelism profile, the implementation of the
            Vectorization profile took a less time.
        \item The Tiling Hyperplane profile is not done by \textit{Substrate} but by
            \textbf{Pluto}~\cite{pluto} : \textit{Substrate} merely use the results of
            \textit{Pluto}'s analyze to get the direction vector of the best Tiling Hyperplane
            \textit{Pluto} could find.

            But adding \textit{Pluto} to the project proved to be more difficult than expect.
            \textit{Pluto} build all its dependences itself, but I wanted to avoid that and
            to have Pluto use my versions of these dependences, but I didn't want to tamper
            to much with its building process. Also, for \textit{Substrate} I've chosen \textbf{cmake}
            as the building system, and \textit{cmake} proved to be difficult to deal with in these
            sort of case.

            In the end, I gave in because I was losing to much time, and I let \textit{Pluto} builds
            its own dependences, even if it meant a lot of useless compilations.
    \end{itemize}
    
    \subsection{Similarity rating algorithms}
    To aggregate 2 statements or meta-statements, they need to be similar enough. We characterize
    this similarity, we propose the following approach :
    \begin{itemize}
        \item We rate the similarity between 0 and 1 for each of the 2 statements' profile types
            (the algorithm are described later).
        \item We use these rate to compute a weighted average.
        \item We compare the result of the weighted average to the minimal rate necessary
            to aggregate and if it is greater or equal to it, we aggregate the statements.
    \end{itemize}
    So the user has to give to the program the weight of each profile type as well as the
    minimal value the weighted average rate need to be to aggregate.


    I choose to go with the weighted average rate because it allow us and the users
    to have a large control over the way \textit{substrate} handles the aggregation.
    Another reason was that with the proliferation of profile types, we didn't have
    any concrete idea to automatically weight the different profile types, so I decide
    to simply give the manual control over it.
        \subsubsection{Data Reuse}
            The Data Reuse Profile is built as the following : for all memory accesses of
            a statement, we first group them by reference, then by Uniformly Generated Set,
            then by reuse direction. The Reuse Profile Rating Algorithm simply count
            the number of accesses of a statement that would belong to the same reuse
            direction group (so also the same reference group and Uniformly Generated Set group)
            of the other rated statement.
            
            In other words, the algorithm counts the accesses in direction group that match
            the same group in the other statement. If a statement has a direction group that doesn't
            have any matching group with the other statement, then the number of access contained
            in this group are not counted.

            We then take this counter and divided it by the total number of accesses of
            the 2 statements.

            For example :
            %TODO graphviz
\begin{lstlisting}[frame=single, language=C, caption={Reuse profile rating example}, label={lst:rp_example}]
for(i=0 ; i<N ; i++)
    for(j=0 ; j<N ; j++)
    {
        A[i][j] = A[i][j+1] + A[i+1][j] + B[i] + B[2*i];    //S1
        C[i][j] = A[i][j+2];                                //S2
    }
\end{lstlisting}
        In this example, S1 would have the following Reuse Profile :
        \begin{center}$
            \{
                \{ 
                    \{A[i][j],A[i][j+1]\},
                    \{A[i+1][j]\}
                \}
            \}
            \{
                \{
                    \{B[i]\}
                \}
                \{
                    \{B[2*i]\}
                \}
            \}$
        \end{center}
            and S2 would have the Reuse Profile :
        \begin{center}$
            \{
                \{
                    \{C[i][j]\}
                \}
            \}
            \{
                \{
                    \{A[i][j+2]\}
                \}
            \}$
        \end{center}
        we can see that the group ${A[i][j],A[i][j+1]}$ of S1 matches the group ${A[i][j+2]}$
        of S2 : so we count $3$. There is no other match between the groups of the 2 statements,
        so the final value of the counter is $3$. The total number accesses is $7$, so the
        Reuse Rate between the two statement is :
        \begin{center}
            $\dfrac{\mathit{Counter}}{\mathit{Total\_number\_accesses}} = \dfrac{3}{7} \approx 0.43$
        \end{center}

        \bigskip

        You can see that even though S2 just have one access in common with S1, we still get
        a "relatively" high rate : this is because S1 has the lot of reuse for a particular
        direction group, and S2 has an access that match that group.

        With a rate of $0.43$, it means that $43\%$ of the 2 statements accesses are
        accessing the same memory area, with the same direction. Which means that if the
        CPU cache is big enough, $43\%$ of the memory accesses lead to \textbf{cache hits}.
            
        \subsubsection{Parallelism}
            Because the parallelism profile of a statement is simply the list of loop
            which could be legally parallelized for the statement, we chose to count the
            loop in common that could be parallelized for both statements, and then
            divide that by the total number of loop that could be parallelized for either one
            or the other statement :
            \begin{center}
                $ \mathrm{Parallelism\_Rate} =  \dfrac{\mathrm{Card}( {PP_1} \cap {PP_2})}{\mathrm{Card}({PP_1} \cup {PP_2})}$
            \end{center}
            with $PP_1$ being the parallelism profile of the first statement, and $PP_2$ the
            one of the second statements.

            \bigskip

            For example :
\begin{lstlisting}[frame=single, language=C, caption={Parallelism profile rating example}, label={lst:pp_example}]
for(i=0 ; i<N ; i++)
    for(j=0 ; j<N ; j++)
        for(k=0 ; k<N ; k++)
            for(l=0 ; l<N ; l++)
            {
                S1(i,j,k,l);
                S2(i,j,k,l);
            }
\end{lstlisting}
        Let say the parallelism profile of S1 and S2 are respectively $\{i,j\}$ and $\{j,k\}$.
        The parallelism rate between those 2 statements would be :
        \begin{center}
            $\dfrac{\mathrm{Card}( \{i,j\} \cap \{j,k\})}{\mathrm{Card}( \{i,j\} \cup \{j,k\})}
            = \dfrac{\mathrm{Card}( \{i\})}{\mathrm{Card}( \{i,j,k\})}
            = \dfrac{1}{3}
            = 0.333...$
        \end{center}

        \subsubsection{Vectorization}
            Because the Vectorization Profile is, like the Parallelism Profile, simply a list of loop, the rating algorithm for the Vectorization Profile is the same as the one for the Parallelism Profiles :
            \begin{center}
                $ \mathrm{Vectorization\_Rate} =  \dfrac{\mathrm{Card}( {VP_1} \cap {VP_2})}{\mathrm{Card}({VP_1} \cup {VP_2})}$
            \end{center}
            with $VP_1$ and $VP_2$ the Vectorization Profile of respectively the first and second rated statements.
        \subsubsection{Tiling Hyperplane}
            Because the Tiling Hyperplane Profile is simply the list of vector
            Pluto has computed for the Hyperplane, we've decided to compute the strict
            equality for the rating algorithm : 
            \begin{center}
                $ \mathrm{Tiling\_Hyperplane\_Rate} =  \left \{
                    \begin{array}{c}
                        0.0\text{, if } \mathit{THP}_1 \neq \mathit{THP}_2. \\
                        1.0\text{, if } \mathit{THP}_1 = \mathit{THP}_2.
                    \end{array} \right.$
            \end{center}
            with $\mathit{THP}_1$ and $\mathit{THP}_2$ the Tiling Hyperplane Profile of 
            respectively the first and second rated statement.

            We are aware that this rating algorithm doesn't allow to precise control using the
            Tiling Hyperplane Profile, because of the "All or Nothing" approach of this
            rating algorithm, but this profile type was the last implemented, and its rating
            algorithm hasn't mature enough yet.
        %\subsubsection{Register Pressure}

    \subsection{Statement Profile Aggregation}
        When we aggregate 2 statements, we also aggregate their different profiles together.
        Here are the different profile aggregation algorithms.
        \subsubsection{Data Reuse}
            In the case of the Data Reuse Profile, we recursively aggregate the Reuse Profiles :
            \begin{itemize}
                \item First we try to compare the Reference groups of the 2 Reuse Profiles :
                    if for the Reference group of a profile there is a Reference group in
                    the other profile that have accesses to the same array reference, then
                    we recursively aggregate these 2 Reference groups. If there is no
                    match, then we just add the Reference group to the new Reuse Profile.
                \item We then do the same for the Uniformly Generated Set groups of the
                    Reference groups we are trying to recursively aggregate. Which means we
                    try to find 2 matching Uniformly Generated Set groups.
                    If that's the case, we try to recursively aggregate them, if not we simply
                    add them to the current Reference group of the new Reuse Profile.
                \item And finally we do the same for the Direction group.
            \end{itemize}
            
            Or in other word, we just fusion matching Direction groups,
            and add the rest non-matching groups without touching them.
        \subsubsection{Parallelism}
            In the case of the Parallelism Profile, we compute the intersection between
            the list of loop from the first and second Parallelism Profile. The result
            of the intersection is the Parallelism Profile of the newly aggregated statement.
        \subsubsection{Vectorization}
            Because the Vectorization Profile is very similar to the Parallelism Profile, 
            the Profile Aggregation Algorithm is the same : the intersection between the
            list of loop from the first and second Vectorization Profile, with the result
            being the Vectorization Profile of the newly aggregated statement.
        \subsubsection{Tiling Hyperplane}
            In the case of the Tiling Hyperplane Profile, because it is so tied to how
            Pluto works, and because Pluto can compute a different Tiling Hyperplane
            of an aggregated statement from the ones it computed when the statements weren't
            aggregated, we decide to let Pluto re-compute the Tiling Hyperplane for the newly
            aggregated statement : we simply delegate it to Pluto.
        %\subsubsection{Register Pressure}

    \subsection{Statement Aggregation Strategy}
        Now that we have defined how we rate the similarity between the profiles of 2 statement,
        it is possible to use the similarity rate to aggregate or not 2 statements, but
        we still need to define an aggregation strategy : should we start from the first
        statement, the last or the one in the middle ? Should we aggregate in that order
        or this order ? Should we introduce some randomness ?

        In the end, we did not have enough time to implement more than one strategy. We
        have some ideas for additional strategies, and they are described in the section
        \ref{sec:future_works}. We will present here the only strategy that is currently implemented
        in \textit{Substrate}.
        \subsubsection{Simple Successive Statements Aggregation}
            The Simple Successive Statements Aggregation Algorithm (SSSAA) iterate through
            the statements of a scop, from the first statement of the scop to the last.
            For each statement S1, it look at the next statement S2 in the scop, and
            check if :
            \begin{itemize}
                \item S1 and S2 share the same Domain Matrix.
                \item S1 and S2 share the same Scattering Matrix, and the same beta depth
                    (i.e. they are in the same loop nest and have the same depth in that nest).
            \end{itemize}
            If these conditions are met, SSSAA then rate the similarity between S1 and S2,
            and if the rate is greater or equal to the minimal rate the user gave to \textit{Substrate}
            then it aggregate S1 and S2 into S3. S3 then take the place of S1, and everything start
            over.

            If the conditions are not met, then S2 take the place of S1 and everything start over.
            

\section{Experiments and Results}
%TODO

    
\section{Conclusions}
The goal of this internship was to study smart ways to aggregate several statements
together to achieve a better performance as well as a better scalability for polyhedral
compilation frameworks.

\bigskip

We've started with the idea that similar or \textit{near behaving} statements should be
aggregated (considered as one instead of two). So we had to come up with a way to characterize
the "similarity" between two statements. For that purpose, we've proposed a model where
each statement is associated with a list of profile built during the analyze pass.
Each profile correspond to a different property a statement can have like \textit{Data Reuse},
\textit{Parallelism} etc...\\
These profiles are a way to quantify or at least measure those properties, and this quantification
allow us to compare the profiles of 2 statements, with numbers. By comparing the profiles
of 2 statements, we are able to compute the Similarity Rate between them, and with that we're
able to characterize the similarity between 2 statements.\\
We have successfully implemented in \textit{Substrate} the profiles and rating algorithms of :
\textit{Data Reuse}, \textit{Parallelism}, \textit{Vectorization}, \textit{Tiling Hyperplane}.

The next things we needed to do was to find Aggregation Strategies. An Aggregation Strategy
determine the way we plan to rate the similarity and aggregate the statements in a scop : in what
order, with or without randomness, with or without human help etc...\\
For now, only one Aggregation Strategy is implemented in \textit{Substrate},
and it is "Simple Successive Statements Aggregation Algorithm" (SSSAA).
SSSAA try to aggregate only successive statements, from the first to the last
statement of the scop, and even though it is a relatively simple algorithm, the results we
are getting by using it are promising.\\
Even though we've seen that our approach is not always applicable, when it does
we have noticed good runtime speedups from some benchmark program, and an overall
speedup of the time taken by \textit{Pluto} to optimize after us in when we were able to
aggregate some statements.

    \subsection{What this internship has brought to me}
        I'm happy to say that I have learn a lot of and about new things during this internship :
        the Polyhedral Model, Data Reuse and Locality, Parallelism and Vectorization detection
        in the Polyhedral Model, Tiling Hyperplane, and some other.\\
        To do that, I had to to read, understand and summarize scientific papers which made
        me a little bit used to it.
        One of my goal was to learn and discover new things about compilation and optimization,
        and this goal is clearly achieved for me.

        I also wanted to know if I had what it takes to work in the Research and if I wanted
        to try to do a PhD thesis about compilation and optimization. This internship gave me
        the opportunities to answer some of my question and even though some of them were
        not answered, I'm still glad to have learned things about myself.\\
        Without this internship, I think I would have never tried to apply for a PhD thesis.

        This internship was also the occasion to practice my programming skills. I didn't
        really learn anything new in C and Python programming, but it felt really nice to
        program in C again, and I had to contribute to some existing projects, and it was
        the first time I was the object of a code review : it was, I think, an important experience.

        An aspect of the internship I didn't expect to do was that I had to schedule myself
        my time and my work : what I needed to do, when I should do it etc... At first I
        was a little bit confused because I did not expect to be \textbf{this} free. But
        I am autonomous by nature so I've quickly adjusted to this situation.

        And finally, there is one more thing I did not expect to do : practice my English skills.\\
        Because the ICPS team has a lot of foreigner, English is used a lot, and it has forced
        me to practice my English, to speak and understand people speaking with different accent.
        And of course I also had to practice my writing with this report.
        I am now more confident about my skills and I really less afraid to speak in English
        anymore : I know more about my strengths and weaknesses in English, and it doesn't
        keep me from trying anymore.


\section{Future Works}
\label{sec:future_works}
At this point in the internship and for this subject, there is still many
questions to answer, problems to solve and many opportunities of improvement.
In this section, we want to mention what ideas we have for future works on \textit{Substrate},
what could be done before the end of the internship, and what would conceivable with more time.
    \subsection{Additional Profiles Types}
        First, we would like to add more Profile Types to the Profile of the statements,
        like \textbf{Register Pressure}. It could be the key to better aggregations and would
        give us a more complete framework to work with for scop optimization.\\

        We are planning to implement the \textbf{Register Pressure} before the end of the
        internship, and to study if this property is significant enough to be used in the
        computation of the \textit{Similarity Rate}.
    \subsection{Additional Aggregation Strategies}
        There is only one Aggregation Strategy implemented in \textit{Substrate}, but
        there is still some things we could try. Were would like to 
        \paragraph{Successive Statements Aggregation with Greedy Algorithm}
            Instead of trying to aggregate the statements from the first to the last
            of the scop being optimized, maybe we should try to use a greedy algorithm
            to chose which successive statements of the scop should be aggregated first : the
            the 2 successive statements that show the best Similarity Rate are aggregated
            first, then we pick the new 2 successive statements with the best Similarity Rate etc...
            until the Similarity Rate between all successive statements are below the minimum rate.\\
        \paragraph{Statements Aggregation using Graphs with Greedy Algorithm}
            The \textit{Successive Statements Aggregation with Greedy Algorithm} only
            try to aggregate successive statements, statements with the same domain,
            the same scattering and the same beta depth, but maybe we could try to
            use the Greedy Algorithm with the Similarity Rate between all statements,
            and try to move and relocate some statements (if the move is legal),
            making them successive, then aggregating them.
        \paragraph{Statements Aggregation using Neural Network}
            We could try to use Neural Network or other Genetic Algorithm, instead
            of Greedy Algorithm, to chose which statement needs to be aggregated to
            maximize the overall similarity of the scop.
    \subsection{Too much different Statement Separator Program}
        One of the goal of this internship was to find a way to aggregate statements
        of a scop to trigger better optimizations, or prevent bad optimizations or even
        reduce the complexity of some optimization.

        But in some cases, it could be beneficial to force 2 statements to be separated
        from each other, because together they prevent some optimization to be applied,
        or they interfere with each other.

        For example, take the following case :
\begin{lstlisting}[frame=single, language=C, caption={Statements Separation example}, label={lst:stmt_sep_example}]
for (i = 0; i < size; i++) {
for (j = 0; j < size; j++) {
    sum += A[i][j]; //S1
    A[i][j] = 0.;   //S2
}
}
\end{lstlisting}
        In this code, S2 could be fully parallelized if alone but not S1, and there is
        some reuse between S1 and S2.

        But if we separate S1 and S2 in 2 different loop nest and fully parallelize S2,
        the resulting code could be faster than the original one. Or maybe not, because
        we would have to trade some reuse for parallelism, and it could be slower.
        
        The question is left open, and that's why we would like to try and answer it, and it
        would be a good trail to follow during a possible PhD thesis.
    \textit{Dynamic Optimization}
        For now, the optimization is done statically and it necessitates the inputs of an
        user. Another approach could be to perform the optimization dynamically, at runtime,
        by collecting information during the execution of the program, and with them
        determining automatically which profiles are the most relevant for this scop, at
        this moment of the execution, etc...\\
        This question would also be a good starting point for a possible PhD thesis.

            
\printglossary

\bibliographystyle{plain}
\bibliography{biblio}
\end{document}
